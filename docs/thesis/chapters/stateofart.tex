\section{State of Art}
Currently the method for automated analysis of body movement consists of an approach involving transferable-utility cooperative games on graph \cite{kolykhalova:2020} that has been summed up below:  
\begin{enumerate}
    \item {A Motion Capture dataset is originated by recording with 13 infra-red cameras. two professional dancers that were equipped with 64 infra-red reflective markers, 5 accelerometers and 1 microphone.} 
    \item{The perceived Origin of Movement are manually annotated by experts in the field.} 
    \item{With the native software “Qualisys Track Manager” have been computed the trajectories of each point.}  
    \item{Joints are created according to the human skeletal structure and then clustered into a set of 20 points.} 
    \item{Then the Shapley value approach is used as solution to the mathematical game built with the weighted vertices.} 
\end{enumerate}
Results are validated with an online survey, where have been asked to users at various level of self-assessed proficiency, 
to select one or two vertices of the skeletal structure as the origin of movement.
They were given hints in three diverse ways by highlighting the joints, either, 
with the highest Shapley values, with the maximum speed, or randomly chosen. 
This study has proven that the most expert users consistently chose as 
origin of movement the one suggested by Shapley values. 

\subsection{Origin of Movement}