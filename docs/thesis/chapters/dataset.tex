\section{Dataset and Marker Set}
A critical phase in our research is the creation of a relevant and high-quality dataset, 
as the quality of the dataset can significantly impact the outcomes of our research efforts.
Here, we detail the technological methodology used to collect the dataset and its constituents.
Our primary objective is to develop an automated method for detecting the initiation of full-body human movements and identifying the most effective predictive movement features.
Central to this research is capturing motion dynamics accurately.
Thus, we require technology for efficient bodily movement capture, like the MoCap.
The participation of these skilled dancers is crucial because their deep knowledge of body mechanics leads to precise and high-quality motion data.
This collaboration yields expressive movements with distinct starting points, aligning with research goals.
Their expertise elevates dataset quality, providing a resource that encapsulates human motion intricacies accurately and artistically.


\subsection{Dataset}
The dataset contains two categories of expressive movements, each associated with a different recording session. \\
The details of both types of movements are provided below.
The dataset comprises <NUMERO FINALE FRAMMENTI> segments recorded using different versions of the \textit{Qualisys Motion Capture system}.
These segments represents various subjects performing simple movement sequences, which are captured from different angles using some cameras.
The videos are synchronized through SMPTE timecode and each video is accompanied by corresponding motion capture data that records the positions of body parts.


\subsubsection{Wholodance Dataset}
This specific portion of the dataset was collected within the framework of the H2020-ICT-2015 EU Project named "WhoLoDance" 
during the month of March in 2016.
These recording sessions are primarily characterized by their multimodal nature, 
meaning that they involve the use of multiple data recording methods. 
The primary focus during these sessions was on capturing expressive aspects of movement, 
specifically qualities related to movement dynamics. These qualities encompassed various factors, 
including the origin of movement, lightness, fluidity, and more.
The participants in these sessions are professional dancers 
who prepared for their recording sessions by designing a series of exercises in advance. 
The recorded movements primarily revolved around contemporary dance techniques. 
Importantly, these movements were performed without any accompanying music to ensure that 
the dancers' performances were not influenced by external auditory cues.

\subsubsection{Montpellier - UniGe Dataset}

\subsection{Dataset Validation with Cohen's Kappa}
TODO ask Oneto
Each fragment of the dataset was classified by us, the thesis students, our professors, and the expert dancer Cora Gasparotti, identifying one or more connections between two joints responsible for the origin of the movement in question.
To measure agreement between validators, we used the Cohen's Kappa $\kappa$, a statistical coefficient commonly used to assess reliability or concordance among human observers or automated systems in the categorization process.
It is generally thought to be a more robust measure than simple percent agreement calculation, as $\kappa$ takes into account the possibility of the agreement occurring by chance.



\subsection{Full Marker Set}

\begin{figure}[H]
    \centering
    \includegraphics[width=0.9\textwidth]{graphics/full_marker_set.png}
    \caption{Full Marker Set of 64 markers}
    \label{fig:full_marker_set}
\end{figure}


\subsection{Reduced Marker Set}
The reduced marker set is achieved by reducing the count of markers employed from the full marker set.
This because a simplified skeletal framework can adeptly communicate essential insights about expressive movements.
The reduction of multiple markers into individual joints is implemented to mitigate the possibility of marker omission.
This reduction was performed by computing the $barycentre$ of joints in a cluster as follows:
\begin{enumerate}
    \item Extract the $x$, $y$, $z$ $coordinates$ for each marker of the cluster.
    \item Compute for each of $x$, $y$, $z$ their average, by summing all the $x$-markers together
    (and similarly for $y$ and $z$) and then dividing by the number of markers in the cluster.
\end{enumerate}    

\begin{table}[H]
    \centering
    \begin{tabular}{|c|c|}
        \hline
        \textbf{Reduced Marker Set Labels} & \textbf{Full Marker Set Labels} \\
        \hline
        right\_foot & RHEL, RMT10, RMT5 \\
        left\_foot & LHEL, LMT10, LMT5 \\
        right\_ankle & RANK \\
        left\_ankle & LANK \\
        right\_knee & RKNE, RKNI \\
        left\_knee & LKNE, LKNI \\
        right\_hip & RFWT, RBWT \\
        hip\_centre & RFWT, LFWT, RBWT, LBWT \\
        left\_hip & LFWT, LBWT \\
        spine & C6, T5, T10, BWT \\
        right\_hand & RPLM, RTHMB, RMID, RPNKY\\
        left\_hand & LPLM, LTHMB, LMID, LPNKY \\
        right\_wrist & ROWR, RIWR \\
        left\_wrist & LOWR, LIWR \\
        right\_elbow & RELB, RIEL\\
        left\_elbow & LELB, LIEL \\
        right\_shoulder & RSHO \\
        shoulder\_centre & LSHO, RSHO \\
        left\_shoulder & LSHO \\
        head & RBHD, LBHD, LFHD, RFHD, ARIEL \\
        \hline
    \end{tabular}
    \caption{Mapping of the full marker set to the reduced marker set}
    \label{tab:labels_joints}
\end{table}


