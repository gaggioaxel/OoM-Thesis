\chapter{Machine Learning}

\section{Features extraction}
Feature extraction in ML is an essential process that involves transforming raw data into a more suitable format for analysis and model building.
It helps identify and capture the most relevant information within the data while simplifying excessive complexity that could make model training challenging.
This process not only contributes to improving model accuracy but also reduces the risk of overfitting and enhances the ability to generalize knowledge gained to new data. \\
As previously mentioned, our dataset consists of 60 samples, each containing the spatial positions of all 20 skeletal joints at each time step.
From this data, we want to extract the necessary features for ML.
Here we will see al the steps involved in order to achieve the desired features. \\

As a first step, we will calculate the $x$, $y$, and $z$ coordinates of the skeleton's barycenter for each time step. \\
We define the barycenter as the point where all the mass of an object or a system of objects is concentrated. \\
In this context, each joint is considered to have unit mass, meaning that all joints contribute equally to the barycenter. \\
If we represent $n$ as the number of 20 joints, the coordinates of the barycenter are obtained by taking the average of the coordinates of these 20 joints, as we can see in the formula:


\begin{equation}
    \textit{Barycenter} (x, y, z) = \left(\frac{1}{n} \sum_{i=1}^{n} x_i, \frac{1}{n} \sum_{i=1}^{n} y_i, \frac{1}{n} \sum_{i=1}^{n} z_i\right)
    \label{formula:baricentro}
\end{equation}
    
where $x_i$, $y_i$, and $z_i$ represent the $x$, $y$, and $z$ coordinates of the $i$-th joint, respectively.

Subsequently, we calculate the distance between each joint and the barycenter, at each time step, using the Euclidean distance in Formula \ref{formula:distance}. \\
These distances from the barycenter will be normalized for each joint, in order to obtain normalized time series between 0 and 1.

\begin{equation}
    x_{norm} = \frac{{x - min(x)}}{{max(x) - min(x)}}
    \label{formula:normalization}
\end{equation}
    
The choice to calculate the distances between joints and the barycenter for each sample plays a crucial role in ensuring the robustness of our ML approach to variations in scale.
In scenarios where dancers or subjects may have different heights or body proportions, relying solely on absolute joint coordinates could introduce bias into the model.
However, by computing these distances and further normalizing them within the range of 0 to 1, we effectively eliminate the influence of scale variations.

This normalization process not only standardizes the data but also allows the ML model to focus on the relative spatial distribution of joints rather than their absolute positions.
Consequently, our model becomes better equipped to recognize patterns and movements across individuals of varying statures, making it more versatile and applicable in real-world scenarios. \\
Finally, to choose the features to extract, we referred to \cite{oneto:2020} and \cite{sama:2010}.
In Table \ref{tab:ml_features} we can see all the features and their relative function.

\begin{table}[H]
    \centering
    \begin{tabular}{|c|c|}
    \hline
    \textbf{Function} & \textbf{Description} \\
    \hline
    mean & Mean Value \\
    var & Variance \\
    mad & Median Absolute Value \\
    max & Largest Value in Array \\
    min & Smallest Value in Array \\
    sma & Signal Magnitude Area \\
    energy & Average Sum of Squares \\
    iqr & Interquantile Range \\
    entropy & Signal Entropy \\
    correlation & Correlation Coefficient \\
    kurtosis & Signal Kurtosis \\
    skewness & Signal Skewness \\
    maxFreqInd & Largest Frequency Component \\
    argMaxFreqInd & Index Largest Frequency Component \\
    meanFreq & Frequency Signal Weighted Average \\
    skewnessFreq & Frequency Signal Skewness \\
    kurtosisFreq & Frequency Signal Kurtosis \\
    ampSprec & Amplitude Spectrum of the Frequency Signal \\
    angle & Phase Angle of the Frequency Signal \\
    \hline
    \end{tabular}
    \caption{List of measures for computing feature vectors}
\label{tab:ml_features}
\end{table}

\section{Method}

The initial approach to classification involves a straightforward binary classification of the origin of the movement within a specific edge.
If the results had been satisfactory, we would then have opted to evaluate progressively more complex models, eventually leading to the actual classification of the movement's origin from a video segment.

\subsection{Model creation}
We started by addressing the issue of imbalanced data in classification. Imbalanced data occurs when one class significantly outnumbers the other(s), leading to challenges in training a classifier that can effectively distinguish between the classes.
To mitigate this imbalance, a resampling technique known as B-SMOTE is employed (\textit{BS1}). The operational concept is detailed in Section \ref{subsec:borderline}.

Following the resampling of the training data, a RF classifier (\textit{RF1}) is trained on this newly balanced dataset.
To further optimize the classifier's performance, a feature selection process is employed with the goal of identifying the most influential features that contribute to accurate classification.
Initially, the most important features are determined through the training of \textit{RF1} on the resampled data. 
These crucial features are selected based on their significance in the classification task. 
Subsequently, a new RF classifier (\textit{RF2}) is trained using only the selected important features. 
To evaluate the performance of the classifier, a test set is chosen using LOOCV, as detailed in Section \ref{subsec:cross_validation}. 

\begin{table}[H]
    \centering
    \begin{tabular}{|c|c|c|}
    \hline
    \textbf{Model} & \textbf{Hyperparameter} & \textbf{Value} \\
    \hline
    \textit{BS1} & \textit{k}-\textit{neighbors} & 0.4 $\cdot$ count(Less Frequent Class)  \\
    & \textit{m}-\textit{neighbors} & 0.4 $\cdot$ count(Most Frequent Class)  \\
    \hline
    \textit{RF1} & \textit{n}\_\textit{estimators} & 500  \\
    & \textit{max}\_\textit{features} & Default  \\
    \hline
    \textit{RF2} & \textit{n}\_\textit{estimators} & 200  \\
    & \textit{max}\_\textit{features} & All  \\
    \hline
    \end{tabular}
    \caption{Hyperparameters tuned for our application}
    \label{tab:ml_param}
\end{table}

\subsection{Binary questions to the Model}

In our ML model, we posed three different questions regarding the origins of movement. 
Given the small and imbalanced dataset, some questions proved to be more challenging to classify. 
Since our dataset is inherently a multi-class dataset, we performed binary classification by reclassifying the dataset in these ways to validate our model against different questions:

\begin{itemize}

    \item \textbf{Is or is not a specific edge:} In our dataset, the most frequent classification is the edge that links \textit{left\_hand} and \textit{left\_wrist}. 
    Therefore, we categorized all samples where the classification was \textit{left\_hand}-\textit{left\_wrist} as 1, while we labeled all other samples as 0 as in Table \ref{tab:to_be_edge_or_not}. 
    We iteratively asked the model whether the test sample was predicted as that edge or not.  
    
    \item \textbf{Is or is not the upper body:} We reclassified the dataset by setting all samples classified in the upper part of the body to 0 and all samples classified in the lower part of the body to 1 following the schema in Table \ref{tab:top_bottom}.
    
    \item \textbf{Is or is not a specific body part:} We regrouped edges based on their respective body part mapped in Table \ref{tab:body_division_5} and took the most frequent to compare against all the others.
    Therefore, we labeled all samples where the classification was in the \textit{Left Leg} as 1, while we categorized all other samples as 0.

\end{itemize}
These reclassification strategies were employed to address the challenges posed by our imbalanced and small dataset, allowing us to perform binary classification for our research purposes.

\begin{figure}[H]
    \centering
    \begin{minipage}{0.4\linewidth}
        %\centering
        \resizebox{\linewidth}{!}{%
            %\centering
            \begin{tabular}{|c|c|}
                \hline
                \textbf{New Label} & \textbf{Edges} \\
                \hline
                This Edge & left\_hand - left\_wrist \\
                \hline
                Not This Edge & right\_hand - right\_wrist \\
                & right\_wrist - right\_elbow \\
                & right\_elbow - right\_shoulder \\
                & right\_shoulder - shoulder\_center \\
                & left\_elbow - left\_shoulder \\
                & left\_shoulder - shoulder\_center \\
                & shoulder\_center - head \\
                & right\_foot - right\_ankle \\
                & right\_ankle - right\_knee \\
                & right\_knee - right\_hip \\
                & right\_hip - hip\_center \\
                & left\_foot - left\_ankle \\
                & left\_knee - left\_hip \\
                & left\_hip - hip\_center \\
                \hline
            \end{tabular}
        }
        \caption{}
        \label{tab:to_be_edge_or_not}
    \end{minipage}
    \hspace{0.1\linewidth}
    \begin{minipage}{0.43\linewidth}
        %\centering
        \resizebox{\linewidth}{!}{%
            \begin{tabular}{|c|c|}
                \hline
                \textbf{Body Division} & \textbf{Edges} \\
                \hline
                Upper & right\_hand - right\_wrist \\
                & right\_wrist - right\_elbow \\
                & right\_elbow - right\_shoulder \\
                & right\_shoulder - shoulder\_center \\
                & left\_hand - left\_wrist \\
                & left\_elbow - left\_shoulder \\
                & left\_shoulder - shoulder\_center \\
                & shoulder\_center - head \\
                \hline
                Lower & right\_foot - right\_ankle \\
                & right\_ankle - right\_knee \\
                & right\_knee - right\_hip \\
                & right\_hip - hip\_center \\
                & left\_foot - left\_ankle \\
                & left\_knee - left\_hip \\
                & left\_hip - hip\_center \\
                \hline
            \end{tabular}
        }
        \caption{}
        \label{tab:top_bottom}
    \end{minipage}

    \begin{minipage}{0.43\linewidth}
        \centering
        \resizebox{\linewidth}{!}{%
            \begin{tabular}{|c|c|}
                \hline
                \textbf{Body Division} & \textbf{Edges} \\
                \hline
                Head & shoulder\_center - head \\
                \hline
                Right Arm & right\_hand - right\_wrist \\
                & right\_wrist - right\_elbow \\
                & right\_elbow - right\_shoulder \\
                & right\_shoulder - shoulder\_center \\
                \hline
                Left Arm & left\_hand - left\_wrist \\
                & left\_elbow - left\_shoulder \\
                & left\_shoulder - shoulder\_center \\
                \hline
                Right Leg & right\_foot - right\_ankle \\
                & right\_ankle - right\_knee \\
                & right\_knee - right\_hip \\
                & right\_hip - hip\_center \\
                \hline
                Left Leg & left\_foot - left\_ankle \\
                & left\_knee - left\_hip \\
                & left\_hip - hip\_center \\
                \hline
            \end{tabular}
        }
        \caption{}
        \label{tab:body_division_5}
    \end{minipage}
\end{figure}

Note that not every edge of the reduced marker set (\ref{tab:labels_joints}) is included because some edges have not been the ground truth for any sample in our dataset.

\section{Results}

\begin{table}[H]
    \centering
    \begin{tabular}{|>{\centering\arraybackslash}p{2cm}|>{\centering\arraybackslash}p{6cm}|>{\centering\arraybackslash}p{2cm}|>{\centering\arraybackslash}p{2cm}|}
    \hline
    \textbf{Label} & \textbf{Edge} & \textbf{TPR} & \textbf{Accuracy} \\
    \hline
    (a) & left\_hand - left\_wrist  & 66\% & 90\% \\
    \hline
    (b) & shoulder\_center - head  & 14\% & 87\% \\
    \hline
    (c) & right\_elbow - right\_shoulder  & 0\%  & 73\% \\ 
    \hline
    (d) & right\_shoulder - shoulder\_center & 33\% & 88\% \\
    \hline
    (e) & right\_knee - right\_hip  & 20\%  & 88\%\\
    \hline
    (f) & left\_knee - left\_hip  & 60\% & 93\%\\ 
    \hline
    \end{tabular}
    \caption{Metrics of the 6 classes most frequent of the dataset}
    \label{tab:cml_results}
\end{table}

  \begin{table}[H]
    \begin{minipage}[b]{0.1\textwidth}
        \centering
        \renewcommand{\arraystretch}{1.6} % Aumenta lo spazio tra le righe del doppio
        \begin{tabular}{|>{\centering\arraybackslash}p{0.5cm}|>{\centering\arraybackslash}p{0.5cm}|}
        \hline
        48 & 3 \\
        \hline
        3 & 6 \\
        \hline
        \end{tabular}
        \caption*{(a)}
        \label{tab:perm1}
    \end{minipage}
    \hfill
    \begin{minipage}[b]{0.1\textwidth}
        \centering
        \renewcommand{\arraystretch}{1.6} % Aumenta lo spazio tra le righe del doppio
        \begin{tabular}{|>{\centering\arraybackslash}p{0.5cm}|>{\centering\arraybackslash}p{0.5cm}|}
        \hline
        51 & 2 \\
        \hline
        6 & 1 \\
        \hline
        \end{tabular}
        \caption*{(b)}
        \label{tab:perm2}
    \end{minipage}
    \hfill
    \begin{minipage}[b]{0.1\textwidth}
        \centering
        \renewcommand{\arraystretch}{1.6} % Aumenta lo spazio tra le righe del doppio
        \begin{tabular}{|>{\centering\arraybackslash}p{0.5cm}|>{\centering\arraybackslash}p{0.5cm}|}
        \hline
        44 & 9 \\
        \hline
        7 & 0 \\
        \hline
        \end{tabular}
        \caption*{(c)}
        \label{tab:perm3}
    \end{minipage}
    \hfill
    \begin{minipage}[b]{0.1\textwidth}
        \centering
        \renewcommand{\arraystretch}{1.6} % Aumenta lo spazio tra le righe del doppio
        \begin{tabular}{|>{\centering\arraybackslash}p{0.5cm}|>{\centering\arraybackslash}p{0.5cm}|}
        \hline
        51 & 3 \\
        \hline
        4 & 2\\
        \hline
        \end{tabular}
        \caption*{(d)}
        \label{tab:perm3}
    \end{minipage}
    \hfill
    \begin{minipage}[b]{0.1\textwidth}
        \centering
        \renewcommand{\arraystretch}{1.6} % Aumenta lo spazio tra le righe del doppio
        \begin{tabular}{|>{\centering\arraybackslash}p{0.5cm}|>{\centering\arraybackslash}p{0.5cm}|}
        \hline
        52 & 3 \\
        \hline
        4 & 1 \\
        \hline
        \end{tabular}
        \caption*{(e)}
        \label{tab:perm3}
    \end{minipage}
    \hfill
    \begin{minipage}[b]{0.1\textwidth}
        \centering
        \renewcommand{\arraystretch}{1.6} % Aumenta lo spazio tra le righe del doppio
        \begin{tabular}{|>{\centering\arraybackslash}p{0.5cm}|>{\centering\arraybackslash}p{0.5cm}|}
        \hline
        53 & 2 \\
        \hline
        2 & 3 \\
        \hline
        \end{tabular}
        \caption*{(f)}
        \label{tab:perm1}
    \end{minipage}
    \hfill
    \caption{j}
    \label{table:results_ml}
\end{table}