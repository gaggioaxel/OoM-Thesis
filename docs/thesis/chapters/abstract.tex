\section*{\Huge Abstract}
The thesis explores a comprehensive approach for predicting movement origins using machine learning techniques, specifically focused on motion capture technology. It commences by reviewing prior research works in the field and introduces a refined methodology to address the challenges associated with movement origin prediction. The proposed approach involves leveraging concepts from graph theory, smoothing theory, and machine learning theory.

Within the theoretical framework, various methodologies are outlined, such as cosine similarity, smoothing techniques like median filtering and rolling mean, graph theory concepts including weighted degree centrality, and machine learning theories like classifiers, k-nearest neighbors, and random forest classifiers. These theoretical foundations serve as the basis for the subsequent methodology and experimentation.

The research methodology involves multiple phases. It starts with the description of the dataset and marker sets used for analysis, detailing data sources, marker specifications, and annotations. Subsequently, it delves into three major frameworks: the graph-based framework, visualization framework, and machine learning framework. Each framework discusses its methodology, techniques used, and specific processes employed to predict movement origins accurately.

The results and discussions section thoroughly examines the outcomes obtained from each framework, highlighting the successes and limitations encountered during experimentation. This section also provides insights into potential future research directions.