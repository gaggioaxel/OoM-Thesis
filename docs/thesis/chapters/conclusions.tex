\chapter{Conclusions}
In this section, conclusions are drawn regarding the study of OoM and the improved visualization of skeletal clusters.
The original planned work underwent modifications due to the limitations of a reduced and unbalanced dataset, coupled with challenges in finding additional complete and consistent samples to enrich the dataset.
Although it was not always easy, addressing and resolving the issues inherent in this type of research proved to be immensely enriching both personally and professionally.
It posed a challenge and provided a glimpse into what we might have to confront in the future, and we are grateful for the opportunity to have had this experience.

In the following section, a synthesis of the results achieved and potential outcomes is presented.
It particularly focuses on performance analysis in terms of operating conditions and further future developments aimed at improving the obtained results, as outlined in Section \ref{sec:discussion} and Section \ref{sec:future_researches}, respectively.

\section{Discussion}
\label{sec:discussion}
The main objectives of our work were to achieve temporal stability in cluster visualization and to classify the OoM using both the WDG and ML. \\

The MaxWPM algorithm is able to provide a clearer and more consistent visualization of clusters, allowing us to understand which joints are most similar during motion, without the annoying and abrupt color changes. \\

NON SO COSA DIRE SUL WDG. \\

As evident, due to the imbalanced dataset and the limited number of available samples, the ML model proves to be effective only for the most frequent classes.
Multiclass classification, in fact, resulted in minimal outcomes that are not worth addressing.
However, the model strives to maximize accuracy and the number of TN at the expense of TP. \\

An essential contribution of this thesis is the development of a movement repository, which will be accessible for any forthcoming research related to movement analysis.



\section{Future researches}
\label{sec:future_researches}
There are several possible future extensions of this work, that could improve on the method
and yield better results. \\

To improve prediction accuracy, we aim to expand the dataset and ensure balance by incorporating new, intentionally designed movement segments.
This will be done without introducing overly intricate motions that could potentially complicate the movement classification.
\\
To move towards real-time automatic classification, markerless MoCap can be useful.
Achieving real-time classification with a marker-based system is highly unlikely.
Despite markerless systems having lower precision compared to marker-based ones, they can be the only feasible alternative for real-time classification.
\\
A potential future project could involve a study on movement fragments characterized by multiple origins.
This would be valuable in examining, for instance, during the act of catching a ball, whether the movement origin is symmetrically distributed throughout the body or if there is a dominant component compared to the others.
\\
The final extension of our thesis involves the examination of small groups of individuals and the exploration of how coordinated movement patterns emerge within these groups.
In this context, we conceptualize a group as a cohesive entity, similar to an organism.
Instead of treating individual joints as players, we adopt a similar approach within the framework of graph and game theory, but at a higher anatomical level.
Here, the group of individuals collectively functions as a single body, with each person within the group assuming the role of a player.
As a result, our research aims to investigate the concept of movement origin in the context of leadership within these group dynamics.