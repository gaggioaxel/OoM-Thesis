\chapter{Conclusions}


\section{Discussion}
The main objectives of our work were to achieve temporal stability in cluster visualization and to classify the OoM using both the WDC and ML.\\

\underline{Temporal Stabilization and Smoothing Results}:\\
The MaxWPM algorithm is able to provide a clearer and more consistent visualization of clusters, allowing us to understand which joints are most similar during motion, without abrupt color changes. \\
Furthermore, the additional smoothing technique allows to better see the evolution of clusters across a longer timespan also in those cases where in a short amount of time clusters still tend to change too fast.\\

\underline{Weighted Degree Centrality Results}:\\
As we can see in Figure \ref*{fig:wdc_results}, the WDC approach appears to work reasonably well when it comes to classifying sources of movement in the upper part of the body. 
However, it is essential to take these results with caution because the dataset is very small and certainly does not have a statistically significant sample of each class. 
Nevertheless, it remains a good starting point. \\
Furthermore, a refinement of this algorithmic approach can lead to a procedure with much more natural and immediate explainability 
compared to the complexity of reconstructing explainability in a machine learning classification.\\

\underline{Machine Learning Results}:\\
From the results obtained in Table \ref{tab:ml_results_joints}, a fluctuating trend is evident, probably due to the limited and imbalanced data. 
Despite efforts to balance the dataset, as they were tested solely on the majority class among the available ones, whose results can be seen in the confusion matrix in Table \ref{tab:ml_results_cm_edge_1}, 
they were not sufficient to compensate for the lack of data in the other classes. \\
What has been achieved, therefore, is an effectiveness in predicting the majority class effectively and a tendency to balance FN and FP. 
By merging the classes, as done in Q2 and Q3, the fluctuations in the results appear to be somewhat mitigated at the expense of a slight deterioration in overall performance (Table \ref{tab:ml_results_body_parts}).
With an extension of the dataset, better and more robust accuracies can be achieved across all classes.
Furthermore, it would also be possible to perform multiclass classification for recognition of multiple origin of movements from the same movement.\\

