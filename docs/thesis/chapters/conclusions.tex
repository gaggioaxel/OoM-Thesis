\chapter{Conclusions}

\section{Discussion}


\section{Future researches}
There are several possible future extensions of this work, that could improve on the method
and yield better results.\\

To improve prediction accuracy, we aim to expand the dataset and ensure balance by incorporating new, intentionally designed movement segments.
This will be done without introducing overly intricate motions that could potentially complicate the movement classification.
\\
To move towards real-time automatic classification, markerless MoCap can be useful.
Achieving real-time classification with a marker-based system is highly unlikely.
Despite markerless systems having lower precision compared to marker-based ones, they can be the only feasible alternative for real-time classification.
\\
A potential future project could involve a study on movement fragments characterized by multiple origins.
This would be valuable in examining, for instance, during the act of catching a ball, whether the movement origin is symmetrically distributed throughout the body or if there is a dominant component compared to the others.
\\
The final extension of our thesis involves the examination of small groups of individuals and the exploration of how coordinated movement patterns emerge within these groups.
In this context, we conceptualize a group as a cohesive entity, similar to an organism.
Instead of treating individual joints as players, we adopt a similar approach within the framework of graph and game theory, but at a higher anatomical level.
Here, the group of individuals collectively functions as a single body, with each person within the group assuming the role of a player.
As a result, our research aims to investigate the concept of movement origin in the context of leadership within these group dynamics.