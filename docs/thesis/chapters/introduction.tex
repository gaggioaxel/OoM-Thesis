\section{Introduction}
Movement is one of the first complex actions that we learn when we are born. It defines how we interact with the world around us. 

Studies in this field date back to Charles Darwin in 19th century with his research in the relationship between movement and emotions, those that are called depressing do not lead to energetic actions. Pain, fear, and griefs when cause complete exhaustion results in prostration, while excitement of the nervous system is typically expressed through frenetic and vibrant movements \cite{darwin}.  

With the Laban Movement Analysis this field reached a formalization, applied to expressive motion in dance, in terms of what the body is doing, interrelationships within it, quality of the movement, changes in physical shape and harmonic interaction of the movement with the space around. 

In general, the analysis in full-body movement spaced both retroactively and proactively as form of expression to convey emotions. 
When moving our bodies, we are not simply performing a physical shift in space, but we use it also to 
communicate affective expressions to others in a nonverbal way \cite{gelder:2009,kleinsmith:2013,karg:2013}. 
For example, the behaviour behind a “caress” can vary from care to hostility, 
if the origin of that movement is either the wrist, the shoulder or if it involves a complex 
contraction of muscle torques from the arm down to the leg. 
This interconnection between many different body parts, brought us to the leading joint hypothesis \cite{dounskaia:2010} 
which offers a novel interpretation of control of human movements that involve multiple joints 
where the central nervous system organizes the execution of the motion act, with a hierarchical process 
that originates from a leading joint and propagates back to the body part that expresses the action. 
In sports and dance, the junction between effectiveness and efficiency of a movement, while also minimizing injuries on the long-term, 
lays in the awareness and exploitation of this concept. 
In my personal experience as a martial art athlete, understanding the correct motion, practicing and fixing misaligned posture is the key for an efficient energy allocation and the whole performance appearance. For example, the correct execution of a cross punch according to the common knowledge within the discipline consists of a complex twist of the upper body with a final braking involving hips, abdominal and back muscles, and the rear foot’s partial rotation \cite{wasik:2013}. 