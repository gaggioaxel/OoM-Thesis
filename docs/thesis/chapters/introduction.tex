\chapter{Introduction}
The study of analyzing human movement traces its origins back to the 19th century, notably with Charles Darwin's investigations into the connection between movement and its meaning. 
Over the years, the fields of movement analysis and non-verbal communication have made significant progress in deciphering the significance of various movements. 
They have grappled with questions like, “Do movements convey meaning in their own right?" and “How do movements convey meaning?".
Movements and non-verbal expressions hold a unique place in communication as they transcend verbal language.
They contribute to the intricate web of conscious and subconscious communication during interactions \cite{Daly:1988, laffaye:2013}. 

However, while there is a wealth of research exploring how emotions are conveyed through facial expressions and vocal cues, the role of full-body movement and expressive gestures in this context has been somewhat overlooked until recent times.

The need for further investigation arises from the rapid advancements in technology that have made full-body real-time movement analysis more accessible and affordable. 
Humans are adept at interpreting the emotional content of non-verbal cues, which underscores the importance of studying the perception of human motion \cite{samadani:2011}. 
Emotions have been observed to manifest through posture and movements, reinforcing the idea that movement plays a crucial role in how we perceive and understand emotions.

Nonetheless, the challenge lies in determining the correct interpretation of a specific movement. 
The context in which a movement occurs is as vital as the observer's perspective. 
Movement analysis has found applications in various fields, spanning sports, the arts, and scientific research. 
One notable contribution from the realm of non-verbal communication research is the Laban Movement Analysis (LMA). 
It posits that “movement is a psycho-physical process, an outward expression of inward intent" \cite{Groff1995LabanMA}. 
LMA provides a structured notation system for examining the structure and expressiveness of movements, particularly in dance.
Within the framework of LMA, there are four distinct categories of movement elements, and one of these categories holds particular relevance to the research conducted in this thesis. 
The Laban Movement Analysis (LMA) encompasses several crucial aspects within the study of movement:
LMA delves into the differentiation and interactions among different body parts during movements, providing insights into how these components function together.
It serves as a method to articulate both sequences of movements and those that occur simultaneously, offering a comprehensive framework for movement description.
Crucially, within the scope of this thesis, LMA focuses on discerning the sources responsible for initiating movements \cite{zhao2001synthesis}.
In essence, it seeks to define the \textit{origin of movement}.
    
\section{Research context}
Movement is one of the first complex actions that we learn when we are born. 
It defines how we interact with the world around us. 

Studies in this field date back to Charles Darwin in 19th century with his research in the relationship 
between movement and emotions, those that are called depressing do not lead to energetic actions. 
Pain, fear, and griefs when cause complete exhaustion results in prostration, 
while excitement of the nervous system is typically expressed through frenetic and vibrant movements \cite{darwin}.  

With the Laban Movement Analysis this field reached a formalization, applied to expressive motion in dance, 
in terms of what the body is doing, interrelationships within it, quality of the movement, 
changes in physical shape and harmonic interaction of the movement with the space around. 

In general, the analysis in full-body movement spaced both retroactively and proactively as form of expression to convey emotions. 
When moving our bodies, we are not simply performing a physical shift in space, but we use it also to 
communicate affective expressions to others in a nonverbal way \cite{gelder:2009,kleinsmith:2013,karg:2013}. 
For example, the behaviour behind a “caress” can vary from care to hostility, 
if the origin of that movement is either the wrist, the shoulder or if it involves a complex 
contraction of muscle torques from the arm down to the leg. 
This interconnection between many different body parts, brought us to the leading joint hypothesis \cite{dounskaia:2010} 
which offers a novel interpretation of control of human movements that involve multiple joints 
where the central nervous system organizes the execution of the motion act, with a hierarchical process 
that originates from a leading joint and propagates back to the body part that expresses the action. 
In sports and dance, the junction between effectiveness and efficiency of a movement, while also minimizing injuries on the long-term, 
lays in the awareness and exploitation of this concept. 

\section{Origins and technological rise of Motion Capture}
The origin of motion capture can be traced back to the mid-20th century when rudimentary methods were employed, 
often involving manual tracking of key points on a subject's body. 
The introduction of marker-based MoCap in the 1970s marked a significant advancement. 
This technique involved attaching reflective markers to specific body points, 
which were then tracked by cameras to reconstruct the subject's movement in a digital environment. 
\begin{figure}[H]
    \centering
    \includegraphics[width=0.9\textwidth]{bodyMarkersExampleImage.png}
    \includegraphics[width=0.9\textwidth]{bodyMarkersExampleMoCap.png}
    \caption{Video frame and MoCap scene with infrared cameras and markers}
    \label{fig:common}
\end{figure}

However, marker-based MoCap had limitations, including occlusion 
(when markers were hidden from view), inaccuracies, and the need for time-consuming calibration processes. 
These limitations led to the development of markerless MoCap technology. 
Markerless MoCap uses computer vision algorithms to track and reconstruct movement without the need for physical markers. 
This approach relies on complex algorithms that can identify and track features of the human body, 
such as joint positions and skeletal structure, from video footage.
\begin{figure}[H]
    \centering
    \includegraphics[width=0.9\textwidth]{MoCapMarkerlessQualisys.png}
    \caption{Markerless MoCap scene}
    \label{fig:mocap}
\end{figure}

The movie “Avatar," directed by James Cameron and released in December 2009, 
played a pivotal role in showcasing the potential of MoCap-based cgi applied to entertainment. 
The film featured groundbreaking visual effects, 
including the integration of live-action performances with computer-generated characters and environments. 
The production used a combination of marker-based and markerless MoCap techniques to capture the actors' 
performances and translate them into the movements of their virtual counterparts, such as the Na'vi characters.
Additionally, facial MoCap was used to capture nuanced facial expressions, enhancing the realism of the characters.
\begin{figure}[H]
    \centering
    \includegraphics[width=0.9\textwidth]{avatar_markers.jpg}
    \caption[]{The actor Sam Worthington and his Na'vi Jake Sully}
    \label{fig:avatar}
\end{figure}

The film's production involved a mix of traditional animation and advanced MoCap techniques, by giving particular emphasis on the facial expression recognition.
From marker-based recording to detection through advanced sensors and high-definition cameras, we see how technological innovations have made the MoCap process increasingly precise and detailed.
Researchers developed algorithms capable of recognizing and tracking human motion patterns from video data, 
enabling more accurate and efficient MoCap.

\section{Motivation}
While relationships between emotions and facial expressions or voice changes have been widely explored, 
leading to the availability of feasible methods for real-time automatic analysis, 
full-body movement has not been equally investigated. 
Various studies have shown great potential for inferring about emotions and many other human activities \cite{Bachmann:2020, preiler:2023}. 
Being able to automatically analyse the origin of movement could improve human performances, 
prevent injuries, promote physical activity, develop cognitive and motor rehabilitation strategies \cite{piana:2016}. 

For this reason, the research in human movement has branches in various fields of study such as biomechanics and neuroscience \cite{vaessen:2019}, 
experimental psychology, and theories from the arts and humanities \cite{camurri:2016}. 

The progress made in this field still does not allow a complete and robust classification of the origin of movement, in an automated way, 
because this relies mainly on arbitrary thresholds to distinguish between different origins and current state of the body, 
like if it is moving or standing still. 
For example, to recognize the instant when a movement starts it is required to manually tune 
minimum speed values that are difficult to automate and generalize for every context. 

Furthermore in movement recognition there are a lot of mid-level features, like the joint angles,
or the limb trajectories or the body segment coordination, which can be extracted and exploited by a comprehensive algorithm 
that weights every feature in an optimize manner, resulting in improved accuracy over all the possible approaches 
that work on them individually, because it could take into account the possible interactions and dependencies between them. 
An holistic approach in this way could leverage the complementary informations present in each feature by weighting them 
based on their relative importance. 
One last point to take into consideration is that algorithms based on single features could end up in overfitting the data
while the comprehensive one has more generalization capability. 

This research aims to contribute to the design of accurate and robust systems for the automated analysis of the origin of movement
by exploiting both the current techniques of analysis and the emerging ML approaches, 
which have become feasible thanks to advancements in computational capabilities of modern machines. 
