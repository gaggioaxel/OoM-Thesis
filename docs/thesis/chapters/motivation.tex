\chapter{Motivation}

While relationships between emotions and facial expressions or voice changes have been widely explored, 
leading to the availability of feasible methods for real-time automatic analysis, 
full-body movement has not been equally investigated. 
Various studies have shown great potential for inferring about emotions and many other human activities. 
Being able to automatically analyse the origin of movement could improve human performances, 
prevent injuries, promote physical activity, develop cognitive and motor rehabilitation strategies \cite{piana:2016}. 

For this reason, the research in human movement has branches in various fields of study such as biomechanics and neuroscience \cite{vaessen:2019}, 
experimental psychology, and theories from the arts and humanities \cite{camurri:2016}. 

The progress made in this field still do not allow a complete and robust classification of the origin of movement, in an automated way, 
because this relies mainly on arbitrary thresholds to distinguish between different origins and current state of the body, 
like if it’s moving or standing still. 
For example, to recognize the instant when a movement starts it’s required to manually tune 
minimum speed values that are difficult to automate and generalize for every context. 

Furthermore in movement recognition there are a lot of mid-level features, like the joint angles, 
or the limb trajectories or the body segment coordination, which can be extracted and exploited by a comprehensive algorithm 
that weights every feature in an optimize manner, resulting in improved accuracy over all the possible approaches 
that work on them individually, because it could take into account the possible interactions and dependencies between them. 
An holistic approach in this way could leverage the complementary informations present in each feature by weighting them 
based on their relative importance. 
One last point to take into consideration is that algorithms based on single features could end up in overfitting the data
while the comprehensive one has more generalization capability. 

This research aims to contribute to the design of accurate and robust systems for the automated analysis of full-body movement
by exploiting both the current techniques of analysis and the emerging Machine Learning approaches, 
which have become feasible thanks to advancements in computational capabilities of modern machines. 