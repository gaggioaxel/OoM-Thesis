\chapter{Method}

The initial approach to classification involves a straightforward binary classification of the origin of the movement within a specific edge.
If the results had been satisfactory, we would then have opted to evaluate progressively more complex models, eventually leading to the actual classification of the movement's origin from a video segment.

\section{Machine Learning Model}
We started by addressing the issue of imbalanced data in classification. Imbalanced data occurs when one class significantly outnumbers the other(s), leading to challenges in training a classifier that can effectively distinguish between the classes.
To mitigate this imbalance, a resampling technique known as BorderlineSMOTE is employed. The operational concept is detailed in Section \ref{subsec:borderline}.
As \textit{k-neighbors} and \textit{m-neighbors} parameters, the best results have been obtained by setting them as the 40\% of the training data with negative class and of the training data with positive class.

After resampling the training data using BorderlineSMOTE, a Random Forest classifier is trained on the resampled data. 
Random Forest is an ensemble learning technique that combines multiple decision trees to improve classification performance. 
In this case, the classifier is trained with 500 decision trees.

\section{Classify if an edge is or is not the OoM}

\section{Classify if a part of the body is or is not the OoM}

\section{Classify if the OoM is on the top or on the bottom of the skeletal}



