\chapter{Method}

The initial approach to classification involves a straightforward binary classification of the origin of the movement within a specific edge.
If the results had been satisfactory, we would then have opted to evaluate progressively more complex models, eventually leading to the actual classification of the movement's origin from a video segment.

\section{Machine Learning Model}
We started by addressing the issue of imbalanced data in classification. Imbalanced data occurs when one class significantly outnumbers the other(s), leading to challenges in training a classifier that can effectively distinguish between the classes.
To mitigate this imbalance, a resampling technique known as B-SMOTE is employed (\textit{BS1}). The operational concept is detailed in Section \ref{subsec:borderline}.

Following the resampling of the training data, a RF classifier (\textit{RF1}) is trained on this newly balanced dataset.
To further optimize the classifier's performance, a feature selection process is employed with the goal of identifying the most influential features that contribute to accurate classification.
Initially, the most important features are determined through the training of \textit{RF1} on the resampled data. 
These crucial features are selected based on their significance in the classification task. 
Subsequently, a new RF classifier (\textit{RF2}) is trained using only the selected important features. 
To evaluate the performance of the classifier, a test set is chosen using LOOCV, as detailed in Section \ref{subsec:cross_validation}. 

\begin{table}[H]
    \centering
    \begin{tabular}{|c|c|c|}
    \hline
    \textbf{Model} & \textbf{Hyperparameter} & \textbf{Value} \\
    \hline
    \textit{BS1} & \textit{k}-\textit{neighbors} & 0.4 $\cdot$ count(Less Frequent Class)  \\
    & \textit{m}-\textit{neighbors} & 0.4 $\cdot$ count(Most Frequent Class)  \\
    \hline
    \textit{RF1} & \textit{n}\_\textit{estimators} & 500  \\
    & \textit{max}\_\textit{features} & Default  \\
    \hline
    \textit{RF2} & \textit{n}\_\textit{estimators} & 200  \\
    & \textit{max}\_\textit{features} & All  \\
    \hline
    \end{tabular}
    \caption{Hyperparameters tuned for our application}
    \label{tab:ml_param}
\end{table}

\section{Binary questions to the Model}

In our ML model, we posed three different questions regarding the origins of movement. 
Given the small and imbalanced dataset, some questions proved to be more challenging to classify. 
Since our dataset is inherently a multi-class dataset, we performed binary classification by reclassifying the dataset in these ways to validate our model against different questions:

\begin{itemize}

    \item \textbf{Edge VS All:} In our dataset, the most frequent classification is the edge that links \textit{left\_hand} and \textit{left\_wrist}. 
    Therefore, we categorized all samples where the classification was "left hand-left elbow" as 1, while we labeled all other samples as 0 as in Table \ref{tab:to_be_edge_or_not}. 
    We iteratively asked the model whether the test sample was predicted as that edge or not.  
    
    \item \textbf{Lower VS Upper:} We reclassified the dataset by setting all samples classified in the upper part of the body to 0 and all samples classified in the lower part of the body to 1 following the schema in Table \ref{tab:top_bottom}.
    
    \item \textbf{Body Part VS All} We regrouped edges based on their respective body part mapped in Table \ref{tab:body_division_5} and took the most frequent to compare against all the others.
    Therefore, we labeled all samples where the classification was in the \textit{right leg} as 1, while we categorized all other samples as 0.

\end{itemize}
These reclassification strategies were employed to address the challenges posed by our imbalanced and small dataset, allowing us to perform binary classification for our research purposes.

\begin{figure}[H]
    \centering
    \begin{minipage}{0.4\linewidth}
        %\centering
        \resizebox{\linewidth}{!}{%
            %\centering
            \begin{tabular}{|c|c|}
                \hline
                \textbf{New Label} & \textbf{Edges} \\
                \hline
                This Edge & left\_hand - left\_wrist \\
                \hline
                Not This Edge & right\_hand - right\_wrist \\
                & right\_wrist - right\_elbow \\
                & right\_elbow - right\_shoulder \\
                & right\_shoulder - shoulder\_center \\
                & left\_elbow - left\_shoulder \\
                & left\_shoulder - shoulder\_center \\
                & shoulder\_center - head \\
                & right\_foot - right\_ankle \\
                & right\_ankle - right\_knee \\
                & right\_knee - right\_hip \\
                & right\_hip - hip\_center \\
                & left\_foot - left\_ankle \\
                & left\_knee - left\_hip \\
                & left\_hip - hip\_center \\
                \hline
            \end{tabular}
        }
        \caption{}
        \label{tab:to_be_edge_or_not}
    \end{minipage}
    \hspace{0.1\linewidth}
    \begin{minipage}{0.43\linewidth}
        %\centering
        \resizebox{\linewidth}{!}{%
            \begin{tabular}{|c|c|}
                \hline
                \textbf{Body Division} & \textbf{Edges} \\
                \hline
                Upper & right\_hand - right\_wrist \\
                & right\_wrist - right\_elbow \\
                & right\_elbow - right\_shoulder \\
                & right\_shoulder - shoulder\_center \\
                & left\_hand - left\_wrist \\
                & left\_elbow - left\_shoulder \\
                & left\_shoulder - shoulder\_center \\
                & shoulder\_center - head \\
                \hline
                Lower & right\_foot - right\_ankle \\
                & right\_ankle - right\_knee \\
                & right\_knee - right\_hip \\
                & right\_hip - hip\_center \\
                & left\_foot - left\_ankle \\
                & left\_knee - left\_hip \\
                & left\_hip - hip\_center \\
                \hline
            \end{tabular}
        }
        \caption{}
        \label{tab:top_bottom}
    \end{minipage}

    \begin{minipage}{0.43\linewidth}
        \centering
        \resizebox{\linewidth}{!}{%
            \begin{tabular}{|c|c|}
                \hline
                \textbf{Body Division} & \textbf{Edges} \\
                \hline
                Head & shoulder\_center - head \\
                \hline
                Right Arm & right\_hand - right\_wrist \\
                & right\_wrist - right\_elbow \\
                & right\_elbow - right\_shoulder \\
                & right\_shoulder - shoulder\_center \\
                \hline
                Left Arm & left\_hand - left\_wrist \\
                & left\_elbow - left\_shoulder \\
                & left\_shoulder - shoulder\_center \\
                \hline
                Right Leg & right\_foot - right\_ankle \\
                & right\_ankle - right\_knee \\
                & right\_knee - right\_hip \\
                & right\_hip - hip\_center \\
                \hline
                Left Leg & left\_foot - left\_ankle \\
                & left\_knee - left\_hip \\
                & left\_hip - hip\_center \\
                \hline
            \end{tabular}
        }
        \caption{}
        \label{tab:body_division_5}
    \end{minipage}
\end{figure}

Note that not every edge of the reduced marker set (\ref{tab:labels_joints}) is included because  