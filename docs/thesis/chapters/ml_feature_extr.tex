\section{Machine Learning features extraction}
Feature extraction in ML is an essential process that involves transforming raw data into a more suitable format for analysis and model building.
It helps identify and capture the most relevant information within the data while simplifying excessive complexity that could make model training challenging.
This process not only contributes to improving model accuracy but also reduces the risk of overfitting and enhances the ability to generalize knowledge gained to new data. \\
As previously mentioned, our dataset consists of 60 samples, each containing the spatial positions of all 20 skeletal joints at each time step.
From this data, we want to extract the necessary features for ML.
Here we will see al the steps involved in order to achieve the desired features. \\

As a first step, we will calculate the $x$, $y$, and $z$ coordinates of the skeleton's barycenter for each time step. \\
We define the barycenter as the point where all the mass of an object or a system of objects is concentrated. \\
In this context, each joint is considered to have unit mass, meaning that all joints contribute equally to the barycenter. \\
If we represent $n$ as the number of 20 joints, the coordinates of the barycenter are obtained by taking the average of the coordinates of these 20 joints, as we can see in the formula:


\begin{equation}
    \textit{Barycenter} (x, y, z) = \left(\frac{1}{n} \sum_{i=1}^{n} x_i, \frac{1}{n} \sum_{i=1}^{n} y_i, \frac{1}{n} \sum_{i=1}^{n} z_i\right)
    \label{formula:baricentro}
\end{equation}
    
where $x_i$, $y_i$, and $z_i$ represent the $x$, $y$, and $z$ coordinates of the $i$-th joint, respectively.

Subsequently, we calculate the distance between each joint and the barycenter, at each time step, using the Euclidean distance in Formula \ref{formula:distance}. \\
These distances from the barycenter will be normalized for each joint, in order to obtain normalized time series between 0 and 1.

\begin{equation}
    x_{norm} = \frac{{x - min(x)}}{{max(x) - min(x)}}
    \label{formula:normalization}
\end{equation}
    
The choice to calculate the distances between joints and the barycenter for each sample plays a crucial role in ensuring the robustness of our ML approach to variations in scale.
In scenarios where dancers or subjects may have different heights or body proportions, relying solely on absolute joint coordinates could introduce bias into the model.
However, by computing these distances and further normalizing them within the range of 0 to 1, we effectively eliminate the influence of scale variations.

This normalization process not only standardizes the data but also allows the ML model to focus on the relative spatial distribution of joints rather than their absolute positions.
Consequently, our model becomes better equipped to recognize patterns and movements across individuals of varying statures, making it more versatile and applicable in real-world scenarios. \\

Finally, to choose the features to extract, we referred to \cite{oneto:2020} and \cite{sama:2010}.
In Table \ref{tab:ml_features} we can see all the features and their relative function.

\begin{table}[H]
    \centering
    \begin{tabular}{|c|c|}
    \hline
    \textbf{Function} & \textbf{Description} \\
    \hline
    mean & Mean Value \\
    var & Variance \\
    mad & Median Absolute Value \\
    max & Largest Value in Array \\
    min & Smallest Value in Array \\
    sma & Signal Magnitude Area \\
    energy & Average Sum of Squares \\
    iqr & Interquantile Range \\
    entropy & Signal Entropy \\
    correlation & Correlation Coefficient \\
    kurtosis & Signal Kurtosis \\
    skewness & Signal Skewness \\
    maxFreqInd & Largest Frequency Component \\
    argMaxFreqInd & Index Largest Frequency Component \\
    meanFreq & Frequency Signal Weighted Average \\
    skewnessFreq & Frequency Signal Skewness \\
    kurtosisFreq & Frequency Signal Kurtosis \\
    ampSprec & Amplitude Spectrum of the Frequency Signal \\
    angle & Phase Angle of the Frequency Signal \\
    \hline
    \end{tabular}
    \caption{List of measures for computing feature vectors}
\label{tab:ml_features}
\end{table}
